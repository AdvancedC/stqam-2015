\documentclass{article}
\usepackage{fullpage}
\usepackage{url}

\begin{document}
\title{Software Testing, Quality Assurance and Maintenance\\SE465 (3A), Winter 2015, version 2}
\author{Patrick Lam}
\renewcommand{\today}{}
\maketitle
\vspace*{-2em}

\section*{Brief Overview}
As you have no doubt discovered, software never works right from the
start. A key technique for getting more acceptable software is
testing. Organized testing can help identify problems in software
systems, enabling developers to fix these problems.  This course will
introduce software testing techniques; while it's not my goal
to produce testers, you should at least be conversant
with up-to-date testing methodologies and techniques.

In this class, we will also touch on software maintenance. While we
greatly (over?) emphasize design in engineering school, maintenance
consumes a large fraction of today's software development resources.

\section*{General Information}

\noindent
\begin{tabbing}
  {\bf Course Web Page:} \= \url{http://patricklam.ca/stqam}\\
  github repo:\= \url{git@github.com:patricklam/stqam-2015.git}\\
  other section: \> \url{http://www.ece.uwaterloo.ca/~lintan/courses/testing/} \\
  {\bf Lectures:} \> MWF 14:30-15:20, RCH 302
\end{tabbing}

\noindent
{\bf Instructor:} \\

\noindent
\hspace*{2em} \begin{minipage}{.6\textwidth}
Prof. Patrick Lam\\
Office: DC2597D\\
Office Hours: Wednesdays 10:30-12:30, or by appointment\\
Email: {\tt p.lam@ece.uwaterloo.ca}\\
Phone: Use email instead!

\end{minipage} \\[1em]

\noindent {\bf Lab Instructor:} \\

\noindent
\hspace*{2em} \begin{minipage}{.6\textwidth}
Bernie Roehl\\
Office: E2-2358\\
Email: {\tt broehl@uwaterloo.ca}\\

\end{minipage} \\[1em]


\noindent {\bf Teaching Assistants:} \\

\noindent
\hspace*{2em}\begin{tabular}{l@{\hspace*{3em}}l@{\hspace*{3em}}l}
  Mike Chong&
  DC 2544&
  {\tt mcchong@uwaterloo.ca}\\
  Taiyue Liu&
  DC 2544&
  {\tt t67liu@uwaterloo.ca}\\
  Jeff Luo&
  DC 3546&
  {\tt j36luo@uwaterloo.ca}\\
  Thibaud Lutellier&
  DC 2544&
  {\tt thibaud.lutellier@uwaterloo.ca}\\
  Abdel Tawakol&
  E5-5037&
  {\tt amtawako@uwaterloo.ca}\\
  Song Wang&
  DC 2544&
  {\tt song.wang@uwaterloo.ca}\\
  Edmund Wong&
  DC 2544&
  {\tt e32wong@uwaterloo.ca}\\
  Jinqiu Yang&
  DC 2544&
  {\tt j223yang@uwaterloo.ca}
\end{tabular}

\section*{Course Description}
\paragraph{Objectives}
\begin{itemize}
  \item You will be able to create and evaluate test suites for reasonably-sized
software systems.

  \item You will learn how to use and write tools for software maintenance and
verification (particularly automated testing tools).

  \item You will gain experience with carrying out modifications
to a large pre-existing software package.
\end{itemize}

We hope that you will learn how to test as a developer, thus making
you a better developer. 

\paragraph{Topics.}
We've been trying to make this course more practical.  Some of the
coverage material still follows the book by Ammann and Offutt. I will
supplement the material in the book with some additional material. The
following list is non-exhaustive.

\begin{itemize}
\item Coverage, subsumption and infeasibility
\item Graph Coverage (includes control-flow graphs, path and dataflow testing, state-based testing, call-graph-based testing, path-based testing)
\item Logic Coverage (includes decision tables)
\item Input Space Coverage
\item Syntax-Based Testing (i.e. mutation testing)
\item Testing in Practice, including concurrency, regression testing, automatic testing tools, mock objects, fuzzing
\item Non-testing-based Software Quality Assurance (code reviews, pair
programming, software model checking and verification)
\item Software Maintenance
\end{itemize}

\section*{Reference Material}
Optional textbook:
\begin{quote}
    Paul Ammann and Jeff Offutt. Introduction to Software Testing. Cambridge University Press, 2008.
\end{quote}

\noindent I also strongly recommend the following book:

\begin{quote}
    Andreas Zeller. Why Programs Fail: a Guide to Systematic Debugging. Morgan Kaufmann, 2005.
\end{quote}

\noindent The Zeller book is quite practical and I expect that it will be useful to you in the future as well.


\section*{Evaluation}
This course includes assignments, a midterm, a course project, and a final
examination.

\begin{tabular}{lrl}
3 individual assignments & 20\% & (6 2/3\% each) \\
Course project (in groups, up to 3/group) & 15\% \\
Midterm & 15\% \\
Final exam & 50\% \\
\end{tabular}

\noindent The midterm and final exams will be open-book, open-notes. 

\paragraph{Schedule.} Assignment handin will be done through the git server at {\tt ecgit.uwaterloo.ca}.\\[-1em]
\begin{center}
\begin{tabular}{ll}
January 14	&A1 out\\
February 2	&A1 due\\
February 4  &A2 out\\
February 23	&A2 due\\
March 2 	&Midterm (7-9PM)\\
March 4     &A3 out\\
March 23	&A3 due\\
April 6 	&Last lecture; project due\\
Exam period	&Final exam
\end{tabular}~\\
\end{center}
You can also find the dates on the following Google calendar:\\
\hspace*{3em}\url{bgt2ebdab4eff0ip7b8flgnah8@group.calendar.google.com}

\paragraph{Group work.} The project will be done in groups.
You may discuss assignments with others, but I expect each of
you to do the assignment independently. I will follow UW's Policy 71
if I discover any cases of plagiarism. I will not use turnitin.

\paragraph{Lateness.} You have 2 days of lateness to use on assignment 
submissions throughout the term. Each day you hand in an assignment
late consumes one of the days of lateness. If you consume all of your
late days, assignments that are still late will get 0. Missed assignments get 0.
e.g. you may hand in A1 one day late and A2 one day late if
you hand in A3 on time.  Or you can hand in A1-A2 on time and
A3 two days late.

\section*{Differences between sections}
Prof. Lin Tan is teaching 2 other sections of this class. We will
cover similar material. The deliverables and exams will be
the same.  You can do the project with partners in other sections.
Lecture material may differ. If that happens, then we would have
exams where you have some choice of which question to answer.

\section*{Required inclusions}
\small \vspace*{-1em}
Academic Integrity: In order to maintain a culture of academic
integrity, members of the University of Waterloo community are
expected to promote honesty, trust, fairness, respect and
responsibility. [Check \url{www.uwaterloo.ca/academicintegrity/} for more
  information.]

\noindent
Grievance: A student who believes that a decision affecting some
aspect of his/her university life has been unfair or unreasonable may
have grounds for initiating a grievance. Read Policy 70, Student
Petitions and Grievances, Section 4,
\url{www.adm.uwaterloo.ca/infosec/Policies/policy70.htm}.  When in doubt
please be certain to contact the department’s administrative assistant
who will provide further assistance.

\noindent
Discipline: A student is expected to know what constitutes academic
integrity [check \url{www.uwaterloo.ca/academicintegrity/}] to avoid
committing an academic offence, and to take responsibility for his/her
actions. A student who is unsure whether an action constitutes an
offence, or who needs help in learning how to avoid offences (e.g.,
plagiarism, cheating) or about “rules” for group work/collaboration
should seek guidance from the course instructor, academic advisor, or
the undergraduate Associate Dean. For information on categories of
offences and types of penalties, students should refer to Policy 71,
Student Discipline,
\url{www.adm.uwaterloo.ca/infosec/Policies/policy71.htm}. For typical
penalties check Guidelines for the Assessment of Penalties,
www.adm.uwaterloo.ca/infosec/guidelines/penaltyguidelines.htm.

\noindent
Appeals: A decision made or penalty imposed under Policy 70 (Student
Petitions and Grievances) (other than a petition) or Policy 71
(Student Discipline) may be appealed if there is a ground. A student
who believes he/she has a ground for an appeal should refer to Policy
72 (Student Appeals)
\url{www.adm.uwaterloo.ca/infosec/Policies/policy72.htm}.  Note for Students
with Disabilities: The Office for Persons with Disabilities (OPD),
located in Needles Hall, Room 1132, collaborates with all academic
departments to arrange appropriate accommodations for students with
disabilities without compromising the academic integrity of the
curriculum. If you require academic accommodations to lessen the
impact of your disability, please register with the OPD at the
beginning of each academic term.


\end{document}
