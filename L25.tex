\documentclass[11pt]{article}
\usepackage{listings}
\usepackage{tikz}
\usepackage{alltt}
\usepackage{hyperref}
\usepackage{url}
%\usepackage{algorithm2e}
\usetikzlibrary{arrows,automata,shapes}
\tikzstyle{block} = [rectangle, draw, fill=blue!20, 
    text width=5em, text centered, rounded corners, minimum height=2em]
\tikzstyle{bt} = [rectangle, draw, fill=blue!20, 
    text width=1em, text centered, rounded corners, minimum height=2em]

\newtheorem{defn}{Definition}
\newtheorem{crit}{Criterion}
\newcommand{\true}{\mbox{\sf true}}
\newcommand{\false}{\mbox{\sf false}}

\newcommand{\handout}[5]{
  \noindent
  \begin{center}
  \framebox{
    \vbox{
      \hbox to 5.78in { {\bf Software Testing, Quality Assurance and Maintenance } \hfill #2 }
      \vspace{4mm}
      \hbox to 5.78in { {\Large \hfill #5  \hfill} }
      \vspace{2mm}
      \hbox to 5.78in { {\em #3 \hfill #4} }
    }
  }
  \end{center}
  \vspace*{4mm}
}

\newcommand{\lecture}[4]{\handout{#1}{#2}{#3}{#4}{Lecture #1}}
\topmargin 0pt
\advance \topmargin by -\headheight
\advance \topmargin by -\headsep
\textheight 8.9in
\oddsidemargin 0pt
\evensidemargin \oddsidemargin
\marginparwidth 0.5in
\textwidth 6.5in

\parindent 0in
\parskip 1.5ex
%\renewcommand{\baselinestretch}{1.25}

%\renewcommand{\baselinestretch}{1.25}
% http://gurmeet.net/2008/09/20/latex-tips-n-tricks-for-conference-papers/
\newcommand{\squishlist}{
 \begin{list}{$\bullet$}
  { \setlength{\itemsep}{0pt}
     \setlength{\parsep}{3pt}
     \setlength{\topsep}{3pt}
     \setlength{\partopsep}{0pt}
     \setlength{\leftmargin}{1.5em}
     \setlength{\labelwidth}{1em}
     \setlength{\labelsep}{0.5em} } }
\newcommand{\squishlisttwo}{
 \begin{list}{$\bullet$}
  { \setlength{\itemsep}{0pt}
     \setlength{\parsep}{0pt}
    \setlength{\topsep}{0pt}
    \setlength{\partopsep}{0pt}
    \setlength{\leftmargin}{2em}
    \setlength{\labelwidth}{1.5em}
    \setlength{\labelsep}{0.5em} } }
\newcommand{\squishend}{
  \end{list}  }
\begin{document}

\lecture{25 --- March 9, 2015}{Winter 2015}{Patrick Lam}{version 1}

\section*{Logic Coverage}

We now shift from graphs to logical expressions, for instance:

\begin{center}
\tt if (visited \&\& x > y || foo(z))
\end{center}

Graphs are made up of nodes, connected by edges. Logical expressions,
or predicates, are made up of clauses, connected by operators.

\paragraph{Motivation.} Logic coverage is required by standard if you're writing
avionics software. The idea is that software makes conditional decisions. While a
condition evaluates to true or false, there are a number of ways that it might
have done so, depending on its subparts. Logic coverage aims to make sure that
test cases explore the subparts adequately.

The standard is called DO-178B. You can find a tutorial about logic coverage,
and in particular the coverage called Modified Condition/Decision Coverage (MC/DC), here:
\url{http://shemesh.larc.nasa.gov/fm/papers/Hayhurst-2001-tm210876-MCDC.pdf}.

\paragraph{Predicates.} A \emph{predicate} is an expression that evaluates
to a logical value. Example:
\[ a \wedge b \leftrightarrow c \]
Here are the operators we allow, in order of precedence (high to low):
\begin{itemize}
\item $\neg$: negation
\item $\wedge$: and (not short-circuit)
\item $\vee$: or (not short-circuit)
\item $\rightarrow$: implication
\item $\oplus$: exclusive or
\item $\leftrightarrow$: equivalence
\end{itemize}
We do not allow quantifiers; they are harder to reason about. Note
also that our operators are not quite the same as the ones in typical
programming languages.

\paragraph{Clauses.} Predicates without logical operators are \emph{clauses};
clauses are, in some sense, ``atomic''. The following predicate contains three clauses:
\begin{center}
{\tt (x > y) || foo(z) \&\& bar}
\end{center}

\paragraph{Logical Equivalence.} Two predicates may be logically equivalent, e.g. 
\[ x \wedge y \vee z \equiv (x \vee z) \wedge (y \vee z), \]
and these predicates are not equivalent to $x \leftrightarrow z$. Equivalence is harder with
short-circuit operators.

\paragraph{Sources of Predicates:} source code, finite state machines, specifications.

\section*{Logic Expression Coverage Criteria}
We'll use the following notation:
\begin{itemize}
\item $P$: a set of predicates;
\item $C$: all clauses making up the predicates of $P$.
\end{itemize}

Let $p \in P$. Then we write $C_p$ for the clauses of the predicate $p$, 
i.e.
\[ C_p = \{ c\ |\ c \in p \}; \qquad C = \bigcup_{p \in P} C_p \]

Given a set of predicates $P$, we might want to cover all of the predicates.
\begin{crit}
{\bf Predicate Coverage} (PC). For each $p \in P$, TR contains two requirements:
1) $p$ evaluates to true; and 2) $p$ evaluates to false.
\end{crit}
PC is analogous to edge coverage on a CFG. (Let $P$ be the 
predicates associated with branches.)

{\sf Example:} ~\\[1em]
\[ P = \{ ( x + 1 == y) \wedge z \}. \]
% predicate coverage: 
% x = 4, y = 5, z = true -> T
% x = 3, y = 5, z = true -> F

PC gives a very coarse-grained view of each predicate.
We can break up predicates into clauses to get more details.

\begin{crit}
{\bf Clause Coverage} (CC). For each $c \in C$, TR contains two requirements:
1) $c$ evaluates to true; 2) $c$ evaluates to false.
\end{crit}

{\sf Example:} ~\\[2em]
\[ (x + 1 == y) \wedge z\] now needs:
% x + 1 == y -> T and F
% z -> T and F
% This gives the tests (x = 4, y = 5, z = T) and (x = 4, y = 4, z = F).

\paragraph{Subsumption.} Are there subsumption relationships between 
CC and PC?\\[3em]

% no, example: p = a \wedge b, C = \{ a, b \}
% T_c = \{ \langle a:T, b:F \rangle, \langle a:F, b:T \rangle \} satisfies CC but not PC;
% T_p = \{ \langle a:T, b:F \rangle, \langle a:F, b:F \rangle \} satisfies PC but not CC.

The obvious exhaustive approach: try everything. (This obviously subsumes
everything else).

\begin{crit}
{\bf Combinatorial Coverage} (CoC). For each $p \in P$, TR has test
requirements for the clauses in $C_p$ to evaluate to each possible
combination of truth values.
\end{crit}

This is also known as multiple condition coverage. Unfortunately, the
number of test requirements, while finite, grows
\underline{\hspace*{7em}} and is hence unscalable.


\section*{A Heuristic: Active Clauses}
So, in general, we can't evaluate every value of every clause, because
there are too many of them. The next best thing is to focus on each
clause and make sure it affects the predicate. That is, we'll test
each clause while it is \emph{active}.

{\sf Example.} Consider the following clause:
\[ p = x \wedge y \vee (z \wedge w) \]
Let's say that we focus on $y$; we'll call it the major clause.  (We
may designate any clause as a major clause at our whim.)  That makes $x, z, w$
minor clauses.

We want to make $y$ \emph{determine} $p$ with certain minor clause
values. That is:
\begin{itemize}
\item if we set $y$ to {\sf true}, then $p$ evaluates to some value $X$;
\item if we set $y$ to {\sf false}, then $p$ must evaluate to $\neg X$.
\end{itemize}

The truth assignment:
\[ x = \mbox{\textvisiblespace} \quad z = \mbox{\textvisiblespace} \quad w = \mbox{\textvisiblespace} \]
will make $y$ determine $p$; in particular, $y$ {\sf true} makes $p$
{\sf true} and $t$ {\sf false} makes $p$ {\sf false}.
% x = true, z = false, w = true.

\begin{defn}
Designate $c_i$ as a major clause in predicate $p$. Then $c_i$ 
\emph{determines} $p$ if the minor clauses $c_j \in p, j \neq i$, have
values such that changing the truth value of $c_i$ changes the truth value
of $p$.
\end{defn}

We do \emph{not} require that $c_i$ has the same truth value as $p$.
{\sf That requirement leads to trouble e.g. for the predicate: }

Informally, determination tests each clause in a context where that
clause has an effect.

\paragraph{Example.}
\[ p = a \vee b \]
{\sf Note that $b$ does not determine $p$ when:}\\[1em]
% a is true

That is, testing $b$ has no effect; the test set {\sf \{ true, false
  \} } does not test $a$ or $b$ effectively.

Here is a variant of clause coverage which uses determination.
\begin{defn}
Active Clause Coverage (ACC). For each $p \in P$ and making each
clause $c_i \in C_p$ major, choose assignments for minor clauses $c_j,
j \neq i$ such that $c_i$ determines $p$. TR has two requirements for each
$c_i$: $c_i$ evaluates to true and $c_i$ evaluates to false.
\end{defn}
This definition is somewhat ambiguous. We'll refine it soon.

\paragraph{Example.} For $p = a \vee b$, make $a$ major. We need
$b$ to be false for $a$ to determine $p$. {\sf This leads to the TRs:} \\[1em]
%\[ \{ \langle a:\true, b:\true \rangle, \langle a:\false, b:\false \rangle \} \]
and similarly for $b$ to determine $p$ {\sf we need TRs:}\\[1em]
%\[ \{ \langle a:\false, b:\true \rangle, \langle a:\false, b:\false \rangle \} \]
Note the overlap between test requirements; it will always exist, meaning
that {\sf our set of TRs for active clause coverage are:}
%\[ \{ \langle a:\true, b:\true \rangle, \langle a:\false, b:\false \rangle, \langle a:\false, b:\true \rangle \} \]

In general, for a predicate with $n$ clauses, we need $n+1$ test requirements.
(It might seem that we'd need $2n$ clauses, but we don't. Why?)


\end{document}

