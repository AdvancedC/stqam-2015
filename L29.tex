\documentclass[11pt]{article}
\usepackage{listings}
\usepackage{tikz}
%\usepackage{algorithm2e}
\usetikzlibrary{arrows,automata,shapes}
\tikzstyle{block} = [rectangle, draw, fill=blue!20, 
    text width=5em, text centered, rounded corners, minimum height=2em]
\tikzstyle{bt} = [rectangle, draw, fill=blue!20, 
    text width=1em, text centered, rounded corners, minimum height=2em]

\newtheorem{defn}{Definition}
\newtheorem{crit}{Criterion}
\newcommand{\true}{\mbox{\sf true}}
\newcommand{\false}{\mbox{\sf false}}

\newcommand{\handout}[5]{
  \noindent
  \begin{center}
  \framebox{
    \vbox{
      \hbox to 5.78in { {\bf Software Testing, Quality Assurance and Maintenance } \hfill #2 }
      \vspace{4mm}
      \hbox to 5.78in { {\Large \hfill #5  \hfill} }
      \vspace{2mm}
      \hbox to 5.78in { {\em #3 \hfill #4} }
    }
  }
  \end{center}
  \vspace*{4mm}
}

\newcommand{\lecture}[4]{\handout{#1}{#2}{#3}{#4}{Lecture #1}}
\topmargin 0pt
\advance \topmargin by -\headheight
\advance \topmargin by -\headsep
\textheight 8.9in
\oddsidemargin 0pt
\evensidemargin \oddsidemargin
\marginparwidth 0.5in
\textwidth 6.5in

\parindent 0in
\parskip 1.5ex
%\renewcommand{\baselinestretch}{1.25}

\begin{document}

\lecture{29 --- March 18, 2015}{Winter 2015}{Patrick Lam}{version 0}

\subsection*{Input Space Partitioning}
Recall the most basic form of testing: feed inputs to the program and
see how it behaves. Of course, we know that we can't feed all inputs
to the program, so we only test a representative set of inputs.
\emph{Input space partitioning} makes this idea more formal: test one
input from each partition.

% picture of dots

{\sf Desirable properties for partitions:}\\[3em]
%% \begin{itemize}
%% \item cover entire domain (completeness)
%% \item must not overlap (disjointness)
%% \end{itemize}

Partitions are usually based on \emph{characteristics} of the input
(or environment),{\sf e.g.}\\[4em]
% colour, length, nullness, implemented interfaces

Here is an example of a bad partition on objects which may implement
interfaces {\tt List} and {\tt Serializable}:
\begin{itemize}
\item objects which implement {\tt List}
\item objects which implement {\tt Serializable}
\item neither
\end{itemize}
What about objects that are serializable lists? Instead,
use the characteristics separately:
\begin{itemize}
\item {\tt List} / not {\tt List}
\item {\tt Serializable} / not {\tt Serializable}
\end{itemize}
Disjoint partitions make it easier to ensure that you indeed cover all
partitions.

\section*{Input Domain Modelling}
We will describe input domains at the unit level, although it applies
equally well at the integration level. Three steps:
\begin{itemize}
\item find units/functions to test;
\item identify parameters of each unit;
\item come up with the model.
\end{itemize}

\paragraph{What Units Should We Test?} First, we determine what to test.

\begin{itemize}
\item In our sample programs below, there's only one testable
  function.
\item For classes in general, could test all public methods (grouping
  them together as needed).
\item Use cases can also give you hints about how to group both
  methods and inputs, or about which methods might be most important.
\end{itemize}

\paragraph{What are the Parameters?} Next, we figure out what inputs the
units might take. {\sf Some possibilities:}\\[4em]

%% \begin{itemize}
%% \item method parameters and other program state;
%% \item fields and data structures
%% \item files
%% \end{itemize}

\paragraph{Create the Model!} Finally, we can group the inputs by finding
characteristics, and creating partitions and blocks, from the values.
Here's an example of the input domain of two digit numbers:

\begin{center}
\begin{tabular}{cccccccccc}
00 & 01 & 02 & 03 & 04 & 05 & 06 & 07 & 08 & 09 \\
10 & 11 & 12 &    & 14 &        & 16 &    & 18 & \\
20 & 21 & 22 &    &    & $\cdots$ &    &    &    & \\
30 & 31 & 32 &    &    & $\cdots$ &    &    &    & \\
$\vdots$ &    &    &    &    & $\cdots$  & & & & $\vdots$ \\
90 &    &    &    &    & $\cdots$  & & & & 99
\end{tabular}
\end{center}

{\sf Possible characteristics and partitions:}\\[6em]
% first digit, last digit, evenness

% partitions by first digit: first digit 0...9
% partitions by evenness: even, odd.

\clearpage
\subsection*{Definitions}
Here are some definitions. 
\begin{itemize}
\item Characteristics: how we distinguish values; 
\item Partition: a way of splitting values into a set of blocks ($p: I \rightarrow \{ 0, 1, \cdots, n \}$)
\item Block: a set of values that are alike with respect to a characteristic ($p^{-1}(k)$)
\end{itemize}
Each input value belongs to one block per characteristic. (We'll talk about
combinations later.)

\subsection*{Input Domain Models}
Coming up with IDMs requires creativity and analysis. Two general approaches:
\begin{itemize}
\item interface-based, using the input space directly; or
\item functionality-based, using a functional or behavioural view of the program.
\end{itemize}

\paragraph{Interface-Based Input Domain Modelling.} Consider each
parameter in isolation. For example: 
\begin{verbatim}
  public boolean containsElement(List list, Object element);
\end{verbatim}
Possible interface-based characteristics:
\begin{itemize}
\item {\tt list} is null; block 1: {\tt true}, block 2: {\tt false}
\item {\tt list} is empty; block 1: {\tt true}, block 2: {\tt false}
\end{itemize}

Notes:
\begin{itemize}
\item (+) ``surprisingly good'', says the book;
\item (+) easy to identify characteristics (but book doesn't provide cookbook);
\item (+) easy to translate to test cases;
\item (-) doesn't use domain knowledge, e.g. relationships between parameters.
\end{itemize}



\end{document}
