\documentclass[11pt]{article}
\usepackage{listings}
\usepackage{tikz}
\usepackage{alltt}
\usepackage{hyperref}
\usepackage{url}
%\usepackage{algorithm2e}
\usetikzlibrary{arrows,automata,shapes}
\tikzstyle{block} = [rectangle, draw, fill=blue!20, 
    text width=5em, text centered, rounded corners, minimum height=2em]
\tikzstyle{bt} = [rectangle, draw, fill=blue!20, 
    text width=1em, text centered, rounded corners, minimum height=2em]

\newtheorem{defn}{Definition}
\newtheorem{crit}{Criterion}
\newtheorem{claim}{Claim}
\newcommand{\strue}{\mbox{\scriptsize \sf true}}
\newcommand{\sfalse}{\mbox{\scriptsize \sf false}}
\newcommand{\true}{\mbox{\sf true}}
\newcommand{\false}{\mbox{\sf false}}

\newcommand{\handout}[5]{
  \noindent
  \begin{center}
  \framebox{
    \vbox{
      \hbox to 5.78in { {\bf Software Testing, Quality Assurance and Maintenance } \hfill #2 }
      \vspace{4mm}
      \hbox to 5.78in { {\Large \hfill #5  \hfill} }
      \vspace{2mm}
      \hbox to 5.78in { {\em #3 \hfill #4} }
    }
  }
  \end{center}
  \vspace*{4mm}
}

\newcommand{\lecture}[4]{\handout{#1}{#2}{#3}{#4}{Lecture #1}}
\topmargin 0pt
\advance \topmargin by -\headheight
\advance \topmargin by -\headsep
\textheight 8.9in
\oddsidemargin 0pt
\evensidemargin \oddsidemargin
\marginparwidth 0.5in
\textwidth 6.5in

\parindent 0in
\parskip 1.5ex
%\renewcommand{\baselinestretch}{1.25}

%\renewcommand{\baselinestretch}{1.25}
% http://gurmeet.net/2008/09/20/latex-tips-n-tricks-for-conference-papers/
\newcommand{\squishlist}{
 \begin{list}{$\bullet$}
  { \setlength{\itemsep}{0pt}
     \setlength{\parsep}{3pt}
     \setlength{\topsep}{3pt}
     \setlength{\partopsep}{0pt}
     \setlength{\leftmargin}{1.5em}
     \setlength{\labelwidth}{1em}
     \setlength{\labelsep}{0.5em} } }
\newcommand{\squishlisttwo}{
 \begin{list}{$\bullet$}
  { \setlength{\itemsep}{0pt}
     \setlength{\parsep}{0pt}
    \setlength{\topsep}{0pt}
    \setlength{\partopsep}{0pt}
    \setlength{\leftmargin}{2em}
    \setlength{\labelwidth}{1.5em}
    \setlength{\labelsep}{0.5em} } }
\newcommand{\squishend}{
  \end{list}  }
\begin{document}

\lecture{26 --- March 11, 2015}{Winter 2015}{Patrick Lam}{version 1}

\paragraph{Determination Example.}
Let's look at one of the examples from last time again.
\[ p : (\neg a \oplus b) \rightarrow c \]
% other example was p : (x \wedge y) \leftrightarrow (z \oplus w)
We construct a truth table as follows:
\[ 
\begin{array}{ccccc}
a & b & \neg a \oplus b & c & p \\
T & T & T & T & T \\
T & T & T & F & F \\
T & T & F & T & F \\
T & F & F & F & T \\
F & T & F & T & F \\
F & T & F & F & T \\
F & F & T & T & T \\
F & F & T & F & F
\end{array}
\]
Note that we use certain patterns to see whether a clause determines $p$ in 
the truth table. For $c$, we check adjacent rows in the truth table.
For $b$, we check pairs of rows separated by one row; and for $a$ we check
one from the top half and one from the bottom half.

\section*{Symbolically Making a Clause Determine a Predicate}
We've seen the brute-force way to figure out when a clause determines
a predicate. It wouldn't scale, since truth tables grow exponentially.
There are also symbolic ways of figuring out when a clause determines
a predicate; here's one way that the textbook presents.

For predicate $p$ and clause $c$,
\begin{itemize}
\item let $p_{c=\strue}$ represent $p$ with $c$ replaced by \strue;
\item and $p_{c=\sfalse}$ represent $p$ with $c$ replaced by \sfalse.
\end{itemize}
Assume that $c$ occurs exactly once in $p$, hence 0 times in $p_{c=\strue}$
and $p_{c=\sfalse}$.

Construct
\[ p_c = p_{c=\strue} \oplus p_{c=\sfalse}. \]

\begin{claim}
$p_c$ describes conditions under which $c$ determine $p$. That is,
if $p_c$ evaluates to \true, then $c$ determines $p$; if $p_c$ evaluates to 
\false, then $c$ does not determine $p$.
\end{claim}
\newpage

Let's look at some examples.

\begin{enumerate}
\item $p = a \vee b$:
\begin{eqnarray*}
p_a = p_{a=\strue} \oplus p_{a=\sfalse} &=& (\true \vee b) \oplus (\false \vee b) \\
&=& \true \oplus b \\
&=& \neg b
\end{eqnarray*}
We see that $a$ determines $p$ precisely when $b$ is false. Symmetrically, $b$ determines $p$ when $a$ is false.
\item $p = a \wedge b$:
\begin{eqnarray*}
p_a = p_{a=\strue} \oplus p_{a=\sfalse} &=& (\true \wedge b) \oplus (\false \wedge b) \\
&=& b \oplus \false \\
&=& b
\end{eqnarray*}
We need $b$ to be \true~ for $a$ to determine $p$; symmetrically for $a$.
\item $p = a \leftrightarrow b$:
\begin{eqnarray*}
p_a = p_{a=\strue} \oplus p_{a=\sfalse} &=& (\true \leftrightarrow b) \oplus (\false \leftrightarrow b) \\
&=& b \oplus \neg b \\
&=& \true
\end{eqnarray*}
In this case, $a$ always determines $p$ regardless of $b$.
\item $p = a \wedge (b \vee c)$:
\begin{eqnarray*}
p_b &=& p_{b=\strue} \oplus p_{b=\sfalse} \\
&=& (a \wedge (\true \vee c)) \oplus (a \wedge (\false \vee c)) \\
&=& (a \wedge \true) \oplus (a \wedge c) = a \oplus (a \wedge c) \\
&=& a \wedge \neg c
\end{eqnarray*}
\end{enumerate}

Why does this construction work? Intuitively:
\begin{itemize}
\item $c$ determines $p$ when setting $c$ to true causes one value of $p$
and setting $c$ to false causes the other value.
\item Everything else should be free to change.
\item Exclusive-or is true when its inputs differ.
\item $p_{c = \strue}$ is basically setting $c$ to \true, $p_{c = \sfalse}$
 is setting $c$ to false.
\end{itemize}

ACC is feasible for a clause $c$ 
if there is a truth assignment to $p$ which makes $p_c$ true.
If $c$ can never determine $p$, then $c$ is redundant with respect to $p$;
that means that we could write $p$ without using $c$, so something is 
probably wrong somewhere.

\section*{Active Clause Coverage Criteria}

\paragraph{Minor Clauses.} The ambiguity in our definition of active clause
coverage stems from our looseness about the minor clauses. All we've
said so far is that the major clause has to determine the predicate
in each test requirement. Consider the following example:
\[ p = a \wedge (b \vee c), \]
and let $a$ be the major clause. Then does the following test suite, 
\[ \langle a:\true, b:{\bf \false}, c:\true \rangle \qquad
\langle a:\false, b:{\bf \true}, c: \true \rangle \]
satisfy ACC? Only on $a$ (but you need more cases for $b$ and $c$).

We will briefly describe three flavours of ACC: General, Correlated and
Restricted. The textbook goes into a lot of detail about these flavours,
but the differences don't seem important in practice.

\begin{crit}
{\bf General Active Clause Coverage} (GACC). For each $p \in P$ and letting
each clause $c_i \in C_p$ be a major clause, choose minor clause values
$c_j$ such that $c_i$ determines $p$. TR contains two test requirements
for each $c_i$: $c_i$ evaluates to true and $c_i$ evaluates to false.
Note that the $c_j$s may be different when $c_i$ evaluates to true and
$c_i$ evaluates to false---there are no restrictions on the $c_j$.
\end{crit}

The above test suite does not satisfy GACC; it does not meet the
requirements for $c$.

Unfortunately, GACC does not subsume PC; it is more
like CC (and obviously subsumes it).

\begin{crit}
{\bf Correlated Active Clause Coverage} (CACC). For each $p \in P$ and
letting each clause $c_i$ be major, choose minor clause values
for $c_j$ such that $c_i$ determines $p$. TR contains two requirements
for each $c_i$: $c_i$ evaluates to true and $c_i$ evaluates to false.
Furthermore, the values chosen for $c_j$ must cause $p$ to be true for
one value of $c_i$ and false for the other.
\end{crit}

To satisfy CACC, you have to satisfy GACC, plus, for each clause,
you must make $p$ true and false.

Last variation: Restricted Active Clause Coverage: ensure that
$c_j$s are always the same for each $c_i$. 

\begin{crit}
{\bf Restricted Active Clause Coverage} (RACC). For each $p \in P$ and
letting each clause $c_i$ be major, choose minor clause values
for $c_j$ such that $c_i$ determines $p$. TR contains two requirements
for each $c_i$: $c_i$ evaluates to true and $c_i$ evaluates to false.
The values for each $c_j$ must be the same when $c_i$ is true and
when $c_i$ is false.
\end{crit}

Key point: must have the same minor clause values in the test set for
each major clause. It is therefore harder to satisfy RACC than CACC.


\paragraph{Example where GACC does not subsume PC.} ~ \\[3em]

% p = a \leftrightarrow b.
% test set <a:T, b:T> (p = T); <a:F, b:F> (p = T) satisfies GACC but not PC.


\end{document}
