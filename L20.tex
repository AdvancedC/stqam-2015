\documentclass[11pt]{article}
\usepackage{listings}
\usepackage{tikz}
\usepackage{alltt}
\usepackage{hyperref}
\usepackage{url}
%\usepackage{algorithm2e}
\usetikzlibrary{arrows,automata,shapes}
\tikzstyle{block} = [rectangle, draw, fill=blue!20, 
    text width=5em, text centered, rounded corners, minimum height=2em]
\tikzstyle{bt} = [rectangle, draw, fill=blue!20, 
    text width=1em, text centered, rounded corners, minimum height=2em]

\newtheorem{defn}{Definition}
\newtheorem{crit}{Criterion}
\newcommand{\true}{\mbox{\sf true}}
\newcommand{\false}{\mbox{\sf false}}

\newcommand{\handout}[5]{
  \noindent
  \begin{center}
  \framebox{
    \vbox{
      \hbox to 5.78in { {\bf Software Testing, Quality Assurance and Maintenance } \hfill #2 }
      \vspace{4mm}
      \hbox to 5.78in { {\Large \hfill #5  \hfill} }
      \vspace{2mm}
      \hbox to 5.78in { {\em #3 \hfill #4} }
    }
  }
  \end{center}
  \vspace*{4mm}
}

\newcommand{\lecture}[4]{\handout{#1}{#2}{#3}{#4}{Lecture #1}}
\topmargin 0pt
\advance \topmargin by -\headheight
\advance \topmargin by -\headsep
\textheight 8.9in
\oddsidemargin 0pt
\evensidemargin \oddsidemargin
\marginparwidth 0.5in
\textwidth 6.5in

\parindent 0in
\parskip 1.5ex
%\renewcommand{\baselinestretch}{1.25}

%\renewcommand{\baselinestretch}{1.25}
% http://gurmeet.net/2008/09/20/latex-tips-n-tricks-for-conference-papers/
\newcommand{\squishlist}{
 \begin{list}{$\bullet$}
  { \setlength{\itemsep}{0pt}
     \setlength{\parsep}{3pt}
     \setlength{\topsep}{3pt}
     \setlength{\partopsep}{0pt}
     \setlength{\leftmargin}{1.5em}
     \setlength{\labelwidth}{1em}
     \setlength{\labelsep}{0.5em} } }
\newcommand{\squishlisttwo}{
 \begin{list}{$\bullet$}
  { \setlength{\itemsep}{0pt}
     \setlength{\parsep}{0pt}
    \setlength{\topsep}{0pt}
    \setlength{\partopsep}{0pt}
    \setlength{\leftmargin}{2em}
    \setlength{\labelwidth}{1.5em}
    \setlength{\labelsep}{0.5em} } }
\newcommand{\squishend}{
  \end{list}  }
\begin{document}

\lecture{20 --- February 25, 2015}{Winter 2015}{Patrick Lam}{version 1}

We've talked about mutation testing in the past few lectures. I thought
I'd summarize some recent research out of Waterloo, in collaboration with
the University of Washington and the University of Sheffield, about: (1) the
effectiveness of mutation testing; and (2) what coverage gets you
in terms of test suite effectiveness.

\section*{Is Mutation Testing Any Good?}
We've talked about mutation testing as a metric for evaluating test
suites and making sure that test suites exercise the system under test
sufficiently. The problem with metrics is that they can be gamed, or
that they might measure not quite the right thing. When using metrics,
it's critical to keep in mind what the right thing is. In this case,
the right thing is the fault detection power of a test suite.

Some researchers set out to determine just that. They carried out a
study, using realistic code, where they isolated a number of bugs, and
evaluated whether or not there exists a correlation between real fault
detection and mutant detection.

\paragraph{Summary.} The answer is {\bf yes}: test suites
that kill more mutants are also better at finding real bugs. The
researchers also investigated when mutation testing fell short---they
enumerated types of bugs that mutation testing, as currently
practiced, would not detect.

\paragraph{Methodology.} The authors used 5 open-source projects.
They isolated a total of 357 reproducible faults in these projects
using the projects' bug reporting systems and source control
repositories. They they generated 230,000 mutants using the
Major mutation framework and investigated the ability of
both developer-written test suites and automatically-generated
test suites (EvoSuite, Randoop, JCrasher) to detect the 357 faults.

For each fault, the authors started with a developer-written test
suite $T_{\mbox{\em \scriptsize bug}}$ that did not detect the fault. Then, using
the source repository, they extracted a developer-written test that
detects the fault. Call this suite $T_{\mbox{\em \scriptsize fix}}$. Does
$T_{\mbox{\em \scriptsize fix}}$ detect more mutants than $T_{\mbox{\em \scriptsize bug}}$?
If so, then we can conclude that the mutant behaves like a bug.

\paragraph{Results.} The authors found that Major-generated mutation
tests could detect 73\% of the faults.  In other words, for 73\% of
faults, some mutant will be killed by a test that also detects the
fault. Increasing mutation coverage thus also increases the likelihood
of finding faults.

The analogous numbers for branch coverage and statement coverage are,
respectively, 50\% and 40\%. Specifically: the 357 tests that find
faults only increase branch coverage 50\% of the time, and they only
increase statement coverage 40\% of the time. So: improving your test
suite often doesn't get rewarded with a better statement coverage
score, and half the time doesn't result in a better branch coverage
score. Conversely, improving statement coverage doesn't help find more bugs because
you're already reaching the fault, but you aren't sensitive to the erroneous state.

The authors also looked at the 27\% of remaining faults that
are not found by mutants. For 10\% of these, better mutation operators
could have helped. The remaining 17\% were not suitable for mutation testing:
they were fixed by e.g. algorithmic improvements or code deletion.

\paragraph{Reference.}
Ren\'e Just, Darioush Jalali, Laura Inozemtseva, Michael D. Ernst,
Reid Holmes, and Gordon Fraser. ``Are Mutants a Valid Substitute for
Real Faults in Software Testing?''  In {\em Foundations of Software
Engineering} 2014. pp654--665.
\url{http://www.linozemtseva.com/research/2014/fse/mutant_validity/}

\section*{What Does (Graph) Coverage Buy You?}
We've talked about graph coverage, notably statement coverage (node
coverage) and branch coverage (edge coverage). They're popular because
they are easy to compute. But, are they any good? Reid Holmes (a
Waterloo CS prof) and his student Laura Inozemtseva set out to answer
that question.

\paragraph{Answer.} Coverage does not correlate with high quality
when it comes to test suites.
Specifically: test suites that are larger are better because
they are larger, not because they have higher coverage.

\paragraph{Methodology.} The authors picked 5 large programs
and created test suites for these programs by taking random
subsets of the developer-written test suites. They measured
coverage and they measured effectiveness (defined as \% mutants
detected; we've seen that detecting mutants is good, above).

\paragraph{Result.} In more technical terms: after controlling
for suite size, coverage is not strongly correlated with effectiveness.

Furthermore, stronger coverage (e.g. branch vs statement, logic vs branch)
doesn't buy you better test suites.

\paragraph{Discussion.} So why are we making you learn about
coverage? Well, it's what's out there, so you should know about it.
But be aware of its limitations.

Plus: if you are not covering some program element, then you obviously
get no information about the behaviour of that element. Low coverage
is bad. But high coverage is not necessarily good.

\paragraph{Reference.} Laura Inozemtseva and Reid Holmes.
``Coverage is Not Strongly Correlated with Test Suite Effectiveness.''
In {\em International Conference on Software Engineering} 2014. pp435--445.
\url{http://www.linozemtseva.com/research/2014/icse/coverage/}

\end{document}
